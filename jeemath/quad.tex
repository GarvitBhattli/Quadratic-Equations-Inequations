\documentclass[journal,12pt,twocolumn]{IEEEtran}
\usepackage{setspace}
\usepackage{gensymb}
\usepackage{caption}
\usepackage{subfiles}
%\usepackage{multirow}
%\usepackage{multicolumn}
%\usepackage{subcaption}
%\doublespacing
\singlespacing
\usepackage{csvsimple}
\usepackage{amsmath}
\usepackage{multicol}
%\usepackage{enumerate}
\usepackage{amssymb}
%\usepackage{graphicx}
\usepackage{newfloat}
%\usepackage{syntax}
\usepackage{listings}
%\usepackage{iithtlc}
\usepackage{color}
\usepackage{tikz}
\usetikzlibrary{shapes,arrows}



%\usepackage{graphicx}
%\usepackage{amssymb}
%\usepackage{relsize}
%\usepackage[cmex10]{amsmath}
%\usepackage{mathtools}
%\usepackage{amsthm}
%\interdisplaylinepenalty=2500
%\savesymbol{iint}
%\usepackage{txfonts}
%\restoresymbol{TXF}{iint}
%\usepackage{wasysym}
\usepackage{amsthm}
\usepackage{mathrsfs}
\usepackage{txfonts}
\usepackage{stfloats}
\usepackage{cite}
\usepackage{cases}
\usepackage{mathtools}
\usepackage{caption}
\usepackage{enumerate}
\usepackage{tfrupee}	
\usepackage{enumitem}
\usepackage{amsmath}
%\usepackage{xtab}
\usepackage{longtable}
\usepackage{multirow}
%\usepackage{algorithm}
%\usepackage{algpseudocode}
\usepackage{enumitem}
\usepackage{mathtools}
\usepackage{hyperref}
%\usepackage[framemethod=tikz]{mdframed}
\usepackage{listings}
    %\usepackage[latin1]{inputenc}                                 %%
    \usepackage{color}                                            %%
    \usepackage{array}                                            %%
    \usepackage{longtable}                                        %%
    \usepackage{calc}                                             %%
    \usepackage{multirow}                                         %%
    \usepackage{hhline}                                           %%
    \usepackage{ifthen}                                           %%
  %optionally (for landscape tables embedded in another document): %%
    \usepackage{lscape}     


\usepackage{url}
\def\UrlBreaks{\do\/\do-}


%\usepackage{stmaryrd}


%\usepackage{wasysym}
%\newcounter{MYtempeqncnt}
\DeclareMathOperator*{\Res}{Res}
%\renewcommand{\baselinestretch}{2}
\renewcommand\thesection{\arabic{section}}
\renewcommand\thesubsection{\thesection.\arabic{subsection}}
\renewcommand\thesubsubsection{\thesubsection.\arabic{subsubsection}}

\renewcommand\thesectiondis{\arabic{section}}
\renewcommand\thesubsectiondis{\thesectiondis.\arabic{subsection}}
\renewcommand\thesubsubsectiondis{\thesubsectiondis.\arabic{subsubsection}}

% correct bad hyphenation here
\hyphenation{op-tical net-works semi-conduc-tor}

%\lstset{
%language=C,
%frame=single, 
%breaklines=true
%}

%\lstset{
	%%basicstyle=\small\ttfamily\bfseries,
	%%numberstyle=\small\ttfamily,
	%language=Octave,
	%backgroundcolor=\color{white},
	%%frame=single,
	%%keywordstyle=\bfseries,
	%%breaklines=true,
	%%showstringspaces=false,
	%%xleftmargin=-10mm,
	%%aboveskip=-1mm,
	%%belowskip=0mm
%}

%\surroundwithmdframed[width=\columnwidth]{lstlisting}
\def\inputGnumericTable{}                                 %%
\lstset{
%language=C,
frame=single, 
breaklines=true,
columns=fullflexible
}
 

\begin{document}
%
\tikzstyle{block} = [rectangle, draw,
    text width=3em, text centered, minimum height=3em]
\tikzstyle{sum} = [draw, circle, node distance=3cm]
\tikzstyle{input} = [coordinate]
\tikzstyle{output} = [coordinate]
\tikzstyle{pinstyle} = [pin edge={to-,thin,black}]

\theoremstyle{definition}
\newtheorem{theorem}{Theorem}[section]
\newtheorem{problem}{Problem}
\newtheorem{proposition}{Proposition}[section]
\newtheorem{lemma}{Lemma}[section]
\newtheorem{corollary}[theorem]{Corollary}
\newtheorem{example}{Example}[section]
\newtheorem{definition}{Definition}[section]
%\newtheorem{algorithm}{Algorithm}[section]
%\newtheorem{cor}{Corollary}
\newcommand{\BEQA}{\begin{eqnarray}}
\newcommand{\EEQA}{\end{eqnarray}}
\newcommand{\define}{\stackrel{\triangle}{=}}

\bibliographystyle{IEEEtran}
%\bibliographystyle{ieeetr}

\providecommand{\nCr}[2]{\,^{#1}C_{#2}} % nCr
\providecommand{\nPr}[2]{\,^{#1}P_{#2}} % nPr
\providecommand{\mbf}{\mathbf}
\providecommand{\pr}[1]{\ensuremath{\Pr\left(#1\right)}}
\providecommand{\qfunc}[1]{\ensuremath{Q\left(#1\right)}}
\providecommand{\sbrak}[1]{\ensuremath{{}\left[#1\right]}}
\providecommand{\lsbrak}[1]{\ensuremath{{}\left[#1\right.}}
\providecommand{\rsbrak}[1]{\ensuremath{{}\left.#1\right]}}
\providecommand{\brak}[1]{\ensuremath{\left(#1\right)}}
\providecommand{\lbrak}[1]{\ensuremath{\left(#1\right.}}
\providecommand{\rbrak}[1]{\ensuremath{\left.#1\right)}}
\providecommand{\cbrak}[1]{\ensuremath{\left\{#1\right\}}}
\providecommand{\lcbrak}[1]{\ensuremath{\left\{#1\right.}}
\providecommand{\rcbrak}[1]{\ensuremath{\left.#1\right\}}}
\theoremstyle{remark}
\newtheorem{rem}{Remark}
\newcommand{\sgn}{\mathop{\mathrm{sgn}}}
\providecommand{\abs}[1]{\left\vert#1\right\vert}
\providecommand{\res}[1]{\Res\displaylimits_{#1}} 
\providecommand{\norm}[1]{\left\Vert#1\right\Vert}
\providecommand{\mtx}[1]{\mathbf{#1}}
\providecommand{\mean}[1]{E\left[ #1 \right]}
\providecommand{\fourier}{\overset{\mathcal{F}}{ \rightleftharpoons}}
%\providecommand{\hilbert}{\overset{\mathcal{H}}{ \rightleftharpoons}}
\providecommand{\system}{\overset{\mathcal{H}}{ \longleftrightarrow}}
	%\newcommand{\solution}[2]{\textbf{Solution:}{#1}}
\newcommand{\solution}{\noindent \textbf{Solution: }}
\newcommand{\myvec}[1]{\ensuremath{\begin{pmatrix}#1\end{pmatrix}}}
\providecommand{\dec}[2]{\ensuremath{\overset{#1}{\underset{#2}{\gtrless}}}}
\DeclarePairedDelimiter{\ceil}{\lceil}{\rceil}
%\numberwithin{equation}{section}
%\numberwithin{problem}{subsection}
%\numberwithin{definition}{subsection}
\makeatletter
\@addtoreset{figure}{section}
\makeatother

\let\StandardTheFigure\thefigure
%\renewcommand{\thefigure}{\theproblem.\arabic{figure}}
\renewcommand{\thefigure}{\thesection}


%\numberwithin{figure}{subsection}

%\numberwithin{equation}{subsection}
%\numberwithin{equation}{section}
%\numberwithin{equation}{problem}
%\numberwithin{problem}{subsection}
\numberwithin{problem}{section}
%%\numberwithin{definition}{subsection}
%\makeatletter
%\@addtoreset{figure}{problem}
%\makeatother
\makeatletter
\@addtoreset{table}{section}
\makeatother

\let\StandardTheFigure\thefigure
\let\StandardTheTable\thetable
\let\vec\mathbf
%%\renewcommand{\thefigure}{\theproblem.\arabic{figure}}
%\renewcommand{\thefigure}{\theproblem}

%%\numberwithin{figure}{section}

%%\numberwithin{figure}{subsection}



\def\putbox#1#2#3{\makebox[0in][l]{\makebox[#1][l]{}\raisebox{\baselineskip}[0in][0in]{\raisebox{#2}[0in][0in]{#3}}}}
     \def\rightbox#1{\makebox[0in][r]{#1}}
     \def\centbox#1{\makebox[0in]{#1}}
     \def\topbox#1{\raisebox{-\baselineskip}[0in][0in]{#1}}
     \def\midbox#1{\raisebox{-0.5\baselineskip}[0in][0in]{#1}}

\vspace{3cm}

\title{ 
%	\logo{
Quadratic Equation and Inequations (Inequalities)
%	}
}

\author{ G V V Sharma$^{*}$% <-this % stops a space
	\thanks{*The author is with the Department
		of Electrical Engineering, Indian Institute of Technology, Hyderabad
		502285 India e-mail:  gadepall@iith.ac.in. All content in this manual is released under GNU GPL.  Free and open source.}
	
}	

\maketitle

%\tableofcontents

\bigskip

\renewcommand{\thefigure}{\theenumi}
\renewcommand{\thetable}{\theenumi}



\begin{enumerate}[label=\arabic*]
\numberwithin{equation}{enumi}
\item The coefficient of $x^{99}$ in the polynomial
$(x-1)(x-2)$.............$(x-100)$ is........

\item If $2+i\sqrt{3}$ is a root of the equation
 $x^{2} +px+q=0$, where p and q are real, then
$(p,q)=(.................... , ....................)$

\item If the product of the roots of the equation
$x^{2}-3kx+2e^{21nk}-1=0$ is $7$, then the roots are real for
$k=.....................$

\item If the quadratic equations $x^{2} +ax+b=0$ and $x^{2}+bx+a=0$ $(a\neq b)$ have a common root then, the numerical value of $a+b$ is....................

\item The solution of equation
$log_7 log_5(\sqrt{x+5}+\sqrt{x})=0$ is .......

\item If $x<0,y<0,x+y+\frac{x} {y}=\frac{1}{2}$ and $(x+y)\frac{x}{y}=-\frac{1}{2}$, then $x=.............$and $y=.................$

\item Let n and k be positive such that $n\geq{\frac{k(k+1)}{2}}$.\\ The number of solutions $(x_1, x_2,.....,x_k), \\x_1\geq{1}, x_2\geq{2},...,x_k\geq{k}$, all integers, satisfying $x_1+x_2+...+x_k=n$, is...........

\item The sum of all the real roots of the equation\\$\begin{vmatrix} x-2 \end{vmatrix}^{2} + \begin{vmatrix} x-1 \end{vmatrix}-2 = 0$ is................

\item For every integer $n>1$, the inequality\\$(n!)^{\frac{1}{n}}<\frac{n+1}{2}$ holds.

\item The equation $2x^{2}+3x+1=0$ has an irrational root.

\item If $a<b<c<d$, then the roots of the equation\\
$(x-a)(x-c)+2(x-b)(x-d)=0$ are real and distinct.

\item If $n_1, n_2,......n_p$ are p positive integers, whose sum is an even number, then the number of odd integers among them is odd.

\item If $P(x)=ax^{2}+bx+c$ and $Q(x)=-ax^{2}+dx+c$, where $ac\neq0$, then $P(x)Q(x)=0$ has at least two real roots.

\item If x and y are positive real numbers and m, n are any positive integers, then $\frac{x^{n}y^{m}}{(1+x^{2n})(1+y^{2m})}>\frac{1}{4}$

\item If l,m,n are real, l $\neq$ m, then the roots by the equation $(l-m)x^{2}-5(l+m)x-2(l-m)=0$ are:
\begin{enumerate}
\item Real and equal
\item Complex
\item Real and unequal
\item None of these
\end{enumerate}
 
\item The equation $x+2y+2z=1$ and $2x+4y+4z=9$ have
\begin{enumerate}
\item Only one solution 
\item Only two solutions
\item Infinite number of solutions
\item None of these
\end{enumerate}
 
\item If x,y and z are real and different and $u=x^{2}+4y^{2}+9z^{2}-6yz-3zx-2xy$ then u is always.
\begin{enumerate}
\item non negative
\item zero
\item non positive
\item none of these
\end{enumerate}

\item Let $a>0,b>0$ and $c>0$. Then the roots of the equation $ax^{2}+bx+c=0$
\begin{enumerate}
\item are real and negative
\item have negative real parts
\item both (a) and(b)
\item none of these
\end{enumerate}

\item Both the roots of the equation $(x-b)(x-c)+(x-a)(x-c)+(x-a)(x-b)=0$ are always
\begin{enumerate}
\item positive
\item real 
\item negative
\item none of these
\end{enumerate}

\item The least value of the expression $2 log_{10}x-log_{x}(0.01)$, for $x>1$, is
\begin{enumerate}
\item $10$
\item $2$ 
\item $-0.01$
\item none of these
\end{enumerate}

\item If $(x^{2}+px+1)$ is a factor of $(ax^{3}+bx+c)$, then
\begin{enumerate}
\item $a^{2}+c^{2}=-ab$
\item $a^{2}-c^{2}=-ab$ 
\item $a^{2}-c^{2}=ab$
\item none of these
\end{enumerate}

\item The number of real solutions of the equation $|x|^{2}-3|x|+2=0$ is
\begin{enumerate}
\item $4$
\item $1$ 
\item $3$
\item $2$ 
\end{enumerate}

\item Two towns A and B are 60 km apart. A school is to be built to serve $150$ students in town A and $50$ students in town B. If the total distance to be travelled by all $200$ students is to be as small as possible, then the school should be built at
\begin{enumerate}
\item town B
\item $45$ km from town A
\item town A
\item $45$km from town B
\end{enumerate}

\item If p,q,r are any real numbers, then
\begin{enumerate}
\item max(p,q)$<$max(p,q,r)
\item min(p,q)$=\frac{1}{2}(p+q-|p-q|)$ 
\item max(p,q)$<$min(p,q,r)
\item none of these 
\end{enumerate}

\item The largest interval for which\\ $x^{12}-x^{9}+x^{4}-x+1>0$ is 
\begin{enumerate}
\item $-4<x\leq0$
\item $0<x<1$ 
\item $-100<x<100$
\item $-\infty<x<\infty$ 
\end{enumerate}

\item The equation $x-\frac{2}{x-1}=1-\frac{2}{x-1}$ has
\begin{enumerate}
\item no root
\item one root 
\item two equal roots
\item infinitely many roots
\end{enumerate}

\item If $a^{2}+b^{2}+c^{2}=1$, then $ab+bc+ca$ lies in the interval
\begin{enumerate}
\item $[\frac{1}{2},2]$
\item $[-1,2]$ 
\item $[-\frac{1}{2},1]$
\item $[-1,\frac{1}{2}]$
\end{enumerate}

\item If $log_{0.3}(x-1)<log_{0.09}(x-1)$, then x lies in the interval
\begin{enumerate}
\item $(2,\infty)$
\item $(1,2)$ 
\item $(-2,-1)$
\item none of these 
\end{enumerate}

\item If $\alpha$ and $\beta$ are the roots of $x^{2}+px+q=0 $ and $ \alpha^{4},\beta^{4}$ are the roots of $x^{2}-rx+s=0$, then the equation $x^{2}-4qx+2q^{2}-r=0$ has always
\begin{enumerate}
\item two real roots
\item two positive roots
\item two negative roots
\item one positive and one negative root 
\end{enumerate}
*Question has more than one correct option.

\item Let a,b,c be real numbers, $a\neq0$. If $\alpha$ is a root of $a^{2}x^{2}+bx+c=0$. $\beta$ is the root of $a^{2}x^{2}-bx-c=0$ and $0<\alpha<\beta$, then the equation $a^{2}x^{2}+2bx+2c=0$ has a root $\gamma$ that always satisfies
\begin{enumerate}
\item $\gamma = \frac{\alpha+\beta}{2}$
\item $\gamma = \alpha+\frac{\beta}{2}$ 
\item $\gamma = \alpha$
\item $\alpha<\gamma<\beta$
\end{enumerate}

\item The number of solutions of the equation $sin(e)^{x}=5^{x}+5^{-x}$ is
\begin{enumerate}
\item $0$
\item $1$ 
\item $2$
\item Infinitely many
\end{enumerate}

\item Let $\alpha, \beta$ be the roots of the equation $(x-a)(x-b)=c, c\neq0$. Then the roots of the equation $(x-\alpha)(x-\beta)+c=0$ are
\begin{enumerate}
\item a,c
\item b,c 
\item a,b
\item $a+c,b+c$
\end{enumerate}

\item The number of points of intersection of two curves $y=2 sinx$ and $y=5x^{2}+2x+3$ is
\begin{enumerate}
\item $0$
\item $1$ 
\item $2$
\item $\infty$ 
\end{enumerate}

\item If p,q,r are $+$ve and are in A.P., the roots of quadratic equation $px^{2}+qx+r=0$ are all real for
\begin{enumerate}
\item $|\frac{r}{p}-7|\geq{4\sqrt{3}}$
\item $|\frac{p}{r}-7|\geq{4\sqrt{3}}$ 
\item all p nd r
\item no p and r 
\end{enumerate}

\item Let p,q$\in\{1,2,3,4\} $. The number of equations of the form $px^{2}+qx+1=0$ having real roots is
\begin{enumerate}
\item $15$
\item $9$ 
\item $7$
\item $8$ 
\end{enumerate}

\item If the roots of the equation\\ $x^{2}-2ax+a^{2}+a-3=0$ are real and less than 3, then
\begin{enumerate}
\item $a<2$
\item $2\leq{a}\leq{3}$ 
\item $3<a\leq{4}$
\item $a>4$ 
\end{enumerate}

\item If $\alpha$ and $\beta$ $(\alpha<\beta)$ are the roots of the equation $x^{2}+bx+c=0$, where $c<0<b$, then
\begin{enumerate}
\item $0<\alpha<\beta$
\item $\alpha<0<\beta<|\alpha|$ 
\item $\alpha<\beta<0$
\item $\alpha<0<|\alpha|<\beta$
\end{enumerate}

\item If a,b,c,d are positive real numbers such that $a+b+c+d=2$, then $M=(a+b)(c+d)$ satisfies the relation 
\begin{enumerate}
\item $0\leq{M}\leq1$
\item $1\leq{M}\leq2$ 
\item $2\leq{M}\leq3$
\item $3\leq{M}\leq4$
\end{enumerate}

\item If $b>a$, then the equation $(x-a)(x-b)-1=0$ has 
\begin{enumerate}
\item both roots in (a,b)
\item both roots in$(-\infty,a)$ 
\item both roots in $(b,+\infty)$
\item one root in $(-\infty,a)$ and the other in $(b,+\infty)$
\end{enumerate}

\item For the equation $3x^{2}+px+3=0, p>0$, if one of the root is square of the other, then p is equal to
\begin{enumerate}
\item $1/3$
\item $1$ 
\item $3$
\item $2/3$
\end{enumerate}

\item If $a_1, a_2,.....,a_n$ are positive real numbers whose product is a fixed number c, then the minimum value of $a_1+a_2+.......+a_{n-1}+2a_n$ is
\begin{enumerate}
\item $n(2c)^{\frac{1}{n}}$
\item $(n+1)c^\frac{1}{n}$ 
\item $2nc^\frac{1}{n}$
\item $(n+1)(2c)^\frac{1}{n}$
\end{enumerate}

\item The set of all real numbers x for which $x^{2}-|x+2|+x>0$, is
\begin{enumerate}
\item $(-\infty,-2)\cup(2,\infty)$
\item $(-\infty,-\sqrt2)\cup(\sqrt2,\infty)$ 
\item $(-\infty,-1)\cup(1,\infty)$
\item $(\sqrt2,\infty)$ 
\end{enumerate}

\item If $\alpha \in(0,\frac{\pi}{2})$ then$\sqrt{x^{2}+x}+\frac{tan^{2}\alpha}{\sqrt{x^{2}+x}}$ is always greater than or equal to 
\begin{enumerate}
\item $2tan\alpha$
\item $1$ 
\item $2$
\item $sec^{2}\alpha$ 
\end{enumerate}

\item For all 'x',$x^{2}+2ax+10-3a>0$, then the interval in which 'a' lies is 
\begin{enumerate}
\item $a<-5$
\item $-5<a<2$ 
\item $a>5$
\item $2<a<5$
\end{enumerate}

\item If one root is square of the other root of the equation $x^{2}+px+q=0$, then the relation between p and q is
\begin{enumerate}
\item $p^{3}-q(3p-1)+q^{2}=0$
\item $p^{3}-q(3p+1)+q^{2}=0$
\item $p^{3}+q(3p-1)+q^{2}=0$
\item $p^{3}+q(3p+1)+q^{2}=0$ 
\end{enumerate}

\item Let a,b,c be the sides of a triangle where ${a}\neq{b}\neq{c}$ and $\lambda \in$R. If the roots of the equation $x^{2}+2(a+b+c)x+3\lambda(ab+bc+ca)=0$ are real, then
\begin{enumerate}
\item $\lambda<\frac{4}{3}$
\item $\lambda>\frac{5}{3}$
\item $\lambda\in(\frac{1}{3},\frac{5}{3})$ 
\item $\lambda\in(\frac{4}{3},\frac{5}{3})$
\end{enumerate}

\item Let $\alpha, \beta$ be the roots of the equation $x^{2}-px+r=0$ and $\frac{\alpha}{2},2\beta$ be the roots of the equation $x^{2}-qx+r=0$. Then the value of r is
\begin{enumerate}
\item $\frac{2}{9}(p-q)(2q-p)$
\item $\frac{2}{9}(q-p)(2p-q)$
\item $\frac{2}{9}(q-2p)(2q-p)$
\item $\frac{2}{9}(2p-q)(2q-p)$ 
\end{enumerate}

\item Let p and q be real numbers such that $p\neq0$, $p^{3}\neq{q}$ and $p^{3}\neq-q$. If $\alpha$ and $\beta$ are non zero complex numbers satisfying $\alpha+\beta=-p$ and $\alpha^{3}+\beta^{3}=q$, then the quadratic equation having $\frac{\alpha}{\beta}$ and $\frac{\beta}{\alpha}$ as its roots is 
\begin{enumerate}
\item $(p^{3}+q)x^{2}-(p^{3}+2q)x+(p^{3}+q)=0$
\item $(p^{3}+q)x^{2}-(p^{3}-2q)x+(p^{3}+q)=0$ 
\item $(p^{3}-q)x^{2}-(5p^{3}-2q)x+(p^{3}-q)=0$
\item $(p^{3}-q)x^{2}-(5p^{3}+2q)x+(p^{3}-q)=0$ 
\end{enumerate}

\item Let $(x_0,y_0)$ be the solution of the following equations $(2x)^{ln2}=(3y)^{ln3}$ and $3^{lnx}=2^{lny}$. Then $x_0$ is
\begin{enumerate}
\item $\frac{1}{6}$
\item $\frac{1}{3}$
\item $\frac{1}{2}$
\item $6$ 
\end{enumerate}

\item Let $\alpha$ and $\beta$ be the roots of $x^{2}-6x-2=0$, with $\alpha>\beta$. If $a_n=\alpha^{n}-\beta^{n}$ for $n\geq1$, then the value of $\frac{a_{10}-2a_{8}}{2a_{9}}$ is
\begin{enumerate}
\item $1$
\item $2$ 
\item $3$
\item $4$ 
\end{enumerate}

\item A value of b for which the equations\\
$x^{2}+bx-1=0$\\
$x^{2}+x+b=0$\\
have one root in common is 
\begin{enumerate}
\item $-\sqrt2$
\item $-i\sqrt3$ 
\item $i\sqrt5$
\item $\sqrt2$ 
\end{enumerate}

\item The quadratic equation $p(x)=0$ with real coefficients has purely imaginary roots. Then the equation p(p(x))=0 has
\begin{enumerate}
\item one purely imaginary root
\item all real roots
\item two real and two purely imaginary roots
\item neither real nor purely imaginary roots
\end{enumerate}

\item Let $-\frac{\pi}{6}<\theta<-\frac{\pi}{12}$. Suppose $\alpha_1$ and $\beta_1$ are the roots of the equation $x^{2}-2x (sec\alpha) +1=0$ and $\alpha_2$ and $\beta_2$ are the roots of the equation $x^{2}-2x tan\theta-1=0$. If $\alpha_1>\beta_1$ and $\alpha_2>\beta_2$, then $\alpha_1+\beta_2$ equals 
\begin{enumerate}
\item $2(sec\theta-tan\theta)$
\item $2sec\theta$ 
\item $-2tan\theta$
\item $0$ 
\end{enumerate}

\item For real x, the function$\frac{(x-a)(x-b)}{x-c}$ will assume all real values provided
\begin{enumerate}
\item $a>b>c$
\item $a<b<c$ 
\item $a>c>b$
\item $a<c<b$ 
\end{enumerate}

\item If S is the set of all real x such that$\frac{2x-1}{2x^{2}+3x^{2}+x}$ is positive, then S contains 
\begin{enumerate}
\item $[-\infty,-\frac{3}{2}]$
\item $[-\frac{3}{2},-\frac{1}{4}]$
\item $[-\frac{1}{4},\frac{1}{2}]$
\item $[\frac{1}{2},3]$
\end{enumerate}

\item If a,b and c are distinct positive numbers, then the expression (b+c-a)(c+a-b)(a+b-c)-abc is
\begin{enumerate}
\item positive
\item negative 
\item non-positive
\item non-negative
\item none of these 
\end{enumerate}

\item If a,b,c,d and p are distinct real numbers such that\\$(a^{2}+b^{2})p^{2}-2(ab+bc+cd)p+(b^{2}+c^{2}+d^{2})\leq0$ then a,b,c,d
\begin{enumerate}
\item are in A.P.
\item are in G.P. 
\item are in H.P.
\item satisfy ab$=$cd
\item none of these 
\end{enumerate}

\item The equation $x^{3/4(log_{2}x)^{2}+log_2x^{-5/4}}=\sqrt2$ has
\begin{enumerate}
\item at least one real solution
\item exactly three solutions
\item exactly one irrational solution
\item complex roots
\end{enumerate}

\item The product of n positive numbers is unity. Then their sum is
\begin{enumerate}
\item a positive integer
\item divisible by n
\item equal to $n+\frac{1}{n}$
\item never less than n
\end{enumerate}

\item Number of divisor of the form $4n+2(n\geq0)$ of the integer $240$ is
\begin{enumerate}
\item $4$
\item $8$ 
\item $10$
\item $3$
\end{enumerate}

\item If $3^{x}=4^{x-1}$, then $x=$
\begin{enumerate}
\item $\frac{2log_32}{2log_32-1}$
\item $\frac{2}{2-2log_23}$
\item $\frac{1}{1-log_43}$
\item $\frac{2log_23}{2log_23-1}$
\end{enumerate}

\item Let S be the set of all non-zero real numbers $\alpha$ such that the quadratic equation $\alpha{x}^{2}-x+\alpha=0$ has two distinct real roots $x_1$ and $x_2$ satisfying the inequality $|x_1-x_2|<1$. Which of the following intervals is (are) a subset (s) of S?
\begin{enumerate}
\item $(-\frac{1}{2},-\frac{1}{\sqrt5})$
\item $(-\frac{1}{\sqrt5},0)$
\item $(0,\frac{1}{\sqrt5})$
\item $(\frac{1}{\sqrt5},\frac{1}{2})$
\end{enumerate}

\item Solve for x: $4^x-3^{x-\frac{1}{2}}=3^{x+\frac{1}{2}}-2^{2x-1}$

\item If (m,n)$=\frac{(1-x^m)(1-x^{m-1}).......(1-x^{m-n+1})}{(1-x)(1-x^2).........(1-x^n)}$ where m and n are positive integers$(n\leq{m})$, show that $(m,n+1)=(m-1,n+1)+x^{m-n-1}(m-1,n)$.

\item Solve for x: $\sqrt{x+1}-\sqrt{x-1}=1$.

\item Solve the following equation for x:
\begin{align} 
2log_{x} a+log_{ax} a+3log_{a^2{x}} a=0, a>0.
\end{align}
\item Show that the square of $\frac{\sqrt{26-15\sqrt{3}}}{5\sqrt{2}-\sqrt{38+5\sqrt{3}}}$,is a rational number.

\item Sketch the solution set of the following system of inequalities: 
\begin{align}
x^{2}+y^{2}-2x\geq0;3x-y-12\leq0;y-x\leq0;y\geq0.
\end{align}

\item Find all integers x for which $(5x-1)<(x+1)^2<(7x-3)$

\item If $\alpha,\beta$ are the roots of $x^2+px+q=0$ and $\gamma,\delta$ are the roots of $x^2+rx+s=0$, evalute$(\alpha-\gamma)(\alpha-\delta)(\beta-\gamma)(\beta-\delta)$ in terms of p,q,,r and s. Deduce the condition that the equations have a common root.

\item Given $n^4<10^n$ for fixed positive integer $n\geq2$ prove that $(n+1)^4<10^{n+1}$

\item Let y=$\sqrt{\frac{(x+1)(x-3)}{(x-2)}}$ Find all the real values of x for which y takes real values.

\item For what values of m, does the system of equations $3x+my=m$,$2x-5y=20$ has solution satisfying the condition $x>0,y>0$.

\item Find the solution set of the system $x+2y+z=1;$ $2x-3y-w=2;$ $x\geq0;y\geq0;z\geq0;w\geq0$.

\item Show that the equation$e^{sinx}-e^{-sinx}-4=0$ has no real solution.

\item mm squares of equal size are arranged to form a rectangle of dimension m by n, where m and n are natural numbers. Two squares will be called 'neighbours' if they have exactly one common side. A natural number is written in each square such that the number written in any square is the arithmetic mean of the numbers written in its neighbouring squares. Show that this is possible only if all the numbers used are equal.

\item If one root of the quadratic equation $ax^2+bx+c=0$ is equal to the n-th power of the other, then show that $(ac)^{\frac{1}{n+1}}+(a^nc)^{\frac{1}{n+1}}+b=0$

\item Find all real values of x which satisfy $x^2-3x+2>0$ and $x^2-2x-4\leq0$.

\item Solve for x;  $(5+2\sqrt{6})^{x^2-3}+(5-2\sqrt6)^{x^2-3}=10$.

\item For $a\leq0$, determine all real roots of the equation$x^2-2a|x-a|-3a^2=0$

\item Find the set of all x for which $\frac{2x}{(2x^2+5x+2)}>\frac{1}{(x+1)}$.

\item Solve ${x^2+4x+3}+2x+5=0$

\item Let a,b,c be real. If $ax^2+bx+c=0$ has two real roots $\alpha$ and $\beta$, where $\alpha<-1$ and $\beta>1$, then show that $1+\frac{c}{a}+|\frac{b}{a}|<0$.

\item Let S be a square of unit area. Consider any quadrilateral which has one vertex on each side of S. If a,b,c and d denote the lengths of the sides of the quadrilateral, prove that $2\leq{a}^2+b^2+c^2+d^2\leq4$.

\item If $\alpha,\beta$ are the roots of $ax^2+bx+c=0, (a\neq0)$ and $\alpha+\delta,\beta+\delta$ are the roots of $Ax^2+Bx+C=0, (A\neq0)$ for some constant $\delta$, then prove that $\frac{b^2-4ac}{a^2}=\frac{B^2-4AC}{A^2}$.

\item Let a,b,c be real numbers with $a\neq0$ and let $\alpha,\beta$ be the roots of the equation $ax^2+bx+c=0$. Express the roots of $a^3x^2+abcx+c^3=0$ in terms of $\alpha,\beta$.

\item If $x^2+(a-b)x+(1-a-b)=0$ where a,b$\in$R then find the values of a for which equation has unequal real roots for all values of b.

\item If a,b,c are positive real numbers. Then prove that $(a+1)^7(b+1)^7(c+1)^7>7^7a^4b^4c^4$.

\item Let a and b be the roots of the equation $x^2-10ax-11b=0$ are c,d then the value of $a+b+c+d$, when $a\neq{b}\neq{c}\neq{d}$, is.

\item Let p,q be integers and let $\alpha,\beta$ be the roots of the equation, $x^2-x-1=0$, where $\alpha\neq\beta$. For n$=0,1,2$,.......,let $a_n=p\alpha^n+q\beta^n$.
FACT: If a and b are rational numbers and \\$a+b\sqrt5=0$, then a$=0=$b.s

\item $a_{12} =$
\begin{enumerate}
\item $a_{11}-a_{10}$
\item $a_{11}+a_{10}$ 
\item $2a_{11}+a_{10}$
\item $a_{11}+2a_{10}$
\end{enumerate}

\item If $a_4 =28$, then $p+2q=\in$
\begin{enumerate}
\item $21$
\item $14$ 
\item $7$
\item $12$ 
\end{enumerate}

\item Let a,b,c,p,q be real numbers. Suppose $\alpha,\beta$ are the roots of the equation $x^2+2px+q=0$ and $\alpha, \frac{1}{\beta}$ are the roots of the equation $ax^2 +2bx+c=0$, where $\beta^2{\notin\lbrace-1,0,1\rbrace}$\\
         STATEMENT-1:$(p^2-q)(b^2-ac)\geq0$
          and\\
         STATEMENT-2:$b\neq{pa}$ or $c\neq{qa}$
\begin{enumerate}
\item STATEMENT-1 is True, STATEMENT-2 is True;STATEMENT-2 is a correct explanation for STATEMENT-1
\item STATEMENT-1 is True,STATEMENT-2 is True;STATEMENT-2 is NOT a correct explanation for STATEMENT-1
\item STATEMENT-1 is True,STATEMENT-2 is False
\item STATEMENT-1 is False,STATEMENT-2 is True
\end{enumerate}

\item Let (x,y,z) be points with integer coordinates satisfying the system of homogeneous equations:
$3x-y-z=0$,$-3x+z=0$,$-3x+2y+z=0$ Then the number of such points for which $x^2+y^2+z^2\leq100$ is

\item The smallest value of k, for which both the roots of the equation $x^2-8kx+16(k^2-k+1)=0$ are real, distinct and have values at least $4$ is

\item The minimum value of the sum of real numbers $a^{-5}, a^{-4}, 3a^{-3}, 1, a^8$ and $a^{10}$ where $a>0$ is

\item The number of distinct real roots of $x^4-4x^3+12x^2+x-1=0$ is

\item If $\alpha\neq\beta$ but $\alpha^2=5\alpha-3$ and $\beta^2=5\beta-3$ then the equation having $\alpha/\beta$ and $\beta/\alpha$ as its roots is 
\begin{enumerate}
\item $3x^2-19x+3=0$
\item $3x^2+19x-3=0$ 
\item $3x^2-19x-3=0$
\item $x^2-5x+3=0$ 
\end{enumerate}

\item Difference between the corresponding roots of $x^2+ax+b=0$ and $x^2+bx+a=0$ is same and $a\neq{b}$, then
\begin{enumerate}
\item $a+b+4=0$
\item $a+b-4=0$ 
\item $a-b-4=0$
\item $a-b+4=0$
\end{enumerate}

\item Product of real roots of the equation $t^2x^2+|x|+9=0$
\begin{enumerate}
\item is always positive
\item is always negative 
\item does not exist
\item none of these 
\end{enumerate}

\item If p and q are the roots of the equation $x^2+px+q=0$, then
\begin{enumerate}
\item $p=1, q=-2$
\item $p=0, q=1$ 
\item $p=-2, q=0$
\item $p=-2, q=1$ 
\end{enumerate}

\item If a,b,c are distinct +ve real numbers and $a^2+b^2+c^2=1$ then $ab+bc+ca$ is
\begin{enumerate}
\item less than 1
\item equal to 1 
\item greater than 1
\item any real number 
\end{enumerate}

\item If the sum of the roots of the quadratic equation $ax^2+bx+c=0$ is equal to the sum of the squares of their reciprocals, then $\frac{a}{c},\frac{b}{a}$ and $\frac{c}{b}$ are in
\begin{enumerate}
\item Arithmetic-Geometric Progression
\item Arithmetic Progression 
\item Geometric Progression
\item Harmonic Progression
\end{enumerate}

\item The value of 'a' for which one root of the quadratic equation $(a^2-5a+3)x^2+(3a-1)x+2=0$ is twice as large as the other is
\begin{enumerate}
\item $-\frac{1}{3}$
\item $\frac{2}{3}$
\item $-\frac{2}{3}$
\item $\frac{1}{3}$
\end{enumerate}

\item The number of real solutions of the equation $x^2-3|x|+2=0$ is
\begin{enumerate}
\item $3$
\item $2$ 
\item $4$
\item $1$
\end{enumerate}

\item The real number x when added to its inverse gives the minimum value of the sum at x equal to
\begin{enumerate}
\item $-2$
\item $2$ 
\item $1$
\item $-1$ 
\end{enumerate}

\item Let two numbers have arithmetic mean $9$ and geometric mean $4$. Then these numbers are the roots of the quadratic equation
\begin{enumerate}
\item $x^2-18x-16=0$
\item $x^2-18x+16=0$ 
\item $x^2+18x-16=0$
\item $x^2+18x+16=0$ 
\end{enumerate}
 
\item If (1-p) is a root of quadratic equation $x^2+px+(1-p)=0$ then its root are
\begin{enumerate}
\item $-1,2$
\item $-1,1$
\item $0,-1$
\item $0,1$
\end{enumerate}

\item If one root of the equation $x^2+px+12=0$ is $4$, while the equation $x^2+px+q=0$ has equal roots, then the value of 'q' is
\begin{enumerate}
\item $4$
\item $12$ 
\item $3$
\item $\frac{49}{4}$ 
\end{enumerate}

\item In a triangle PQR,$\angle{R}=\frac{\pi}{2}$. If tan$(\frac{P}{2})$ and -tan$(\frac{Q}{2})$ are roots of $ax^2+bx+c=0, a\neq0$ then
\begin{enumerate}
\item $a=b+c$
\item $c=a+b$
\item $b=c$
\item $b=a+c$ 
\end{enumerate}

\item If both the roots of the quadratic equation $x^2-2kx+k^2+k-5=0$ are less than $5$, then k lies in the interval
\begin{enumerate}
\item $(5,6]$
\item $(6,\infty)$
\item $(-\infty, 4)$
\item $[4,5]$ 
\end{enumerate}

\item If the roots of the quadratic equation $x^2+px+q=0$ are tan$30\degree$ and tan${15}\degree$, respectively, then the value of 2+q-p is
\begin{enumerate}
\item $2$
\item $3$
\item $0$
\item $1$ 
\end{enumerate}

\item All the values of m for which both roots of the equation $x^2-2mx+m^2-1=0$ are greater than $-2$ but less than $4$, lies in the interval
\begin{enumerate}
\item $-2<m<0$
\item $m>3$
\item $-1<m<3$
\item $1<m<4$ 
\end{enumerate}

\item If x is real , the maximum value of $\frac{3x^2+9x+17}{3x^2+9x+7}$ is
\begin{enumerate}
\item $\frac{1}{4}$
\item $41$
\item $1$
\item $\frac{17}{7}$ 
\end{enumerate}

\item If the difference between the roots of the equation $x^2+ax+1=0$ is less than $\sqrt5$, then the set of possible values of a is
\begin{enumerate}
\item $(3,\infty)$
\item $(-\infty,-3)$
\item $(-3,3)$
\item $(-3,\infty)$
\end{enumerate}

\item Statement-1: For every natural number $n\geq{2}$,
$\frac{1}{\sqrt1}+\frac{1}{\sqrt2}+.............+\frac{1}{\sqrt{n}}>\sqrt{n}$ 
Statement-2: For every natural number $n\geq{2}$,
$\sqrt{n(n+1)}<n+1$
\begin{enumerate}
\item Statement-1 is false, Statement-2 is true
\item Statement-1 is true, Statement-2 is true; Statement-2 is a correct explanation for Statement-1
\item Statement-1 is true, Statement-2 is true; Statement-2 is not a correct explanation for Statement-1
\item Statement-1 is true, Statement-2 is false 
\end{enumerate}

\item The quadratic equation $x^2-6x+a=0$ and $x^2-cx+6=0$ have one root in common. The other roots of the first and second equations are integers in the ratio 4:3. Then the common root is
\begin{enumerate}
\item $1$
\item $4$
\item $3$
\item $2$ 
\end{enumerate}

\item If the roots of the equation $bx^2+cx+a=0$ be imaginary, then for all real values of x, the expression $3b^2x^2+6bcx+2c^2$ is:
\begin{enumerate}
\item less than $4ab$
\item greater than $-4ab$
\item less than $-4ab$
\item greater than $4ab$
\end{enumerate}

\item If $|z-\frac{4}{z}|=2$, then the maximum value of $|Z|$ is equal to: 
\begin{enumerate}
\item $\sqrt{5}+1$
\item $2$
\item $2+\sqrt{2}$
\item $\sqrt{3}+1$ 
\end{enumerate}

\item If $\alpha$ and $\beta$ are the roots of the equation $x^2-x+1=0$, then $\alpha^{2009}+\beta^{2009}=$
\begin{enumerate}
\item $-1$
\item $1$
\item $2$
\item $-2$ 
\end{enumerate}

\item The equation $e^{sinx}-e^{-sinx}-4=0$ has:
\begin{enumerate}
\item infinite number of real roots
\item no real roots
\item exactly one real root
\item exactly four real roots
\end{enumerate}

\item The real number $k$ for which the equation, $2x^3+3x+k=0$ has two distinct real roots in [0,1]
\begin{enumerate}
\item lies between $1$ and $2$
\item lies between $2$ and $3$
\item lies between $-1$ and $0$
\item does not exist 
\end{enumerate}

\item The number of values of $k$, for which the system of equations:\\
$(k+1)x+8y=4k$\\
$kx+(k+3)y=3k-1$
\begin{enumerate}
\item infinite
\item $1$
\item $2$
\item $3$
\end{enumerate}

\item If the equations $x^2+2x+3=0$ and $ax^2+bx+c=0$, a,b,c$\in$R, have a common root, then a:b:c is
\begin{enumerate}
\item $1:2:3$
\item $3:2:1$
\item $1:3:2$
\item $3:1:2$ 
\end{enumerate}

\item Is a$\in$R and the equation $-3(x-[x])^2+2(x-[x])+a^2=0$ 
(where[x] denotes the greatest integer $\leq{x})$ has no integral solution, then all possible values of a lie in the interval:
\begin{enumerate}
\item $(-2,-1)$
\item $(-\infty,-2)\cup(2,\infty)$
\item $(-1,0)\cup(0,1)$
\item $(1,2)$
\end{enumerate}

\item Let $\alpha$ and $\beta$ be the roots of the equation $px^2+qx+r=0,p\neq0$. If $p,q,r$ are in A.P. and $\frac{1}{\alpha}+\frac{1}{\beta}=4$, then the value of $|\alpha-\beta|$ is:
\begin{enumerate}
\item $\frac{\sqrt{34}}{9}$
\item $\frac{2\sqrt{13}}{9}$
\item $\frac{\sqrt{61}}{9}$
\item $\frac{2\sqrt{17}}{9}$
\end{enumerate}

\item Let $\alpha$ and $\beta$ be the roots of equation $x^2-6x-2=0$. If $a_n=\alpha^n-\beta^n$, for $n\geq1$, then the value of $\frac{a_{10}-2a_8}{2a_9}$ is equal to:
\begin{enumerate}
\item $3$
\item $-3$
\item $6$
\item $-6$ 
\end{enumerate}

\item The sum of all real values of x satisfying the equation $(x^2-5x+5)^{x^2+4x-60}=1$ is:
\begin{enumerate}
\item $6$
\item $5$
\item $3$
\item $-4$ 
\end{enumerate}

\item If $\alpha,\beta\in$ C are the distinct roots, of the equation $x^2-x+1=0$, then $\alpha^{101}+\beta^{107}$ is equal to:
\begin{enumerate}
\item $0$
\item $1$
\item $2$
\item $-1$ 
\end{enumerate}

\item Let p,q $\in$R. If $2-\sqrt{3}$ is a root of the quadratic equation, $x^2+px+q=0$, then:
\begin{enumerate}
\item $p^2-4q+12 = 0$
\item $q^2-4p-16 = 0$
\item $q^2+4p+14 = 0$
\item $p^2-4q-12 = 0$
\end{enumerate}